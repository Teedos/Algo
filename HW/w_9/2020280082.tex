\documentclass{article}
\usepackage[utf8]{inputenc} % allow utf-8 input
\usepackage[T1]{fontenc}    % use 8-bit T1 fonts
\usepackage{hyperref}       % hyperlinks
\usepackage{url}            % simple URL typesetting
\usepackage{booktabs}       % professional-quality tables
\usepackage{amsfonts}       % blackboard math symbols
\usepackage{nicefrac}       % compact symbols for 1/2, etc.
\usepackage{microtype}      % microtypography
\usepackage{xcolor}         % colors
\usepackage{graphicx}
\usepackage{amsmath}
\usepackage{float}
\usepackage{lmodern}
\usepackage{algpseudocode}

\title{Algorithms Homework- week 9}

% The \author macro works with any number of authors. There are two commands
% used to separate the names and addresses of multiple authors: \And and \AND.
%
% Using \And between authors leaves it to LaTeX to determine where to break the
% lines. Using \AND forces a line break at that point. So, if LaTeX puts 3 of 4
% authors names on the first line, and the last on the second line, try using
% \AND instead of \And before the third author name.

\author{
	Massimo Hong\\
	Student id: 2020280082\\
 } 

\begin{document}

\maketitle
\section*{1}
The run time complexity of the algorithm is $log(n)$, given that $m = n$.
We only need to consider the operations inside the while loops, as the others cost 1.
\begin{itemize}
	\item The outer loop ($m > 0$): can be executed a maximum of $C < m$ times.
	\item The inner loop ($!(m\%2)$): can be run a maximum of $log(m = n)$ times.
\end{itemize}
We can conclude that an upper bound for the run time complexity is given by $Clog(n) = O(log(n))$, for $n > 0$.
Of course, if $n= 0$, it will have a $O(1)$ cost.

\section*{2}
\begin{equation*}
	2n^2 - 3n = O(n^2)
\end{equation*}
With $n \geq 0$ and $k = 1$:
\begin{equation*}
	\frac{f(n)}{g(n)} = \frac{2n^2 - 3n}{n^2} \leq \frac{2n^2}{n^2} = 2
\end{equation*}
$C = 2$
\begin{equation*}
	2n^2 - 3n \leq 2n^2 = O(n^2)
\end{equation*}
\section*{3}
\begin{enumerate}
	\item The input of the Majority Element problem is the same as the input of a Sorting problem: An array of elements.
	\item We already know several algorithms to solve a sorting problem, i.e. Insertion sort, with a time complexity of $\theta(nlogn)$.
	\item The output of the sorting problem is an ordered array A. The majority element (if it exists) is the element at the position of $\lfloor\frac{A.length}{2}\rfloor$.
\end{enumerate}
\section*{4}
\begin{algorithmic}
	\Procedure{Linear search}{$A, v$}
		\For {$i = 1$ to $A.length$}
			\If {A[i] == v}
				\State \Return $i$
			\EndIf
		\EndFor
		\State \Return NIL
	\EndProcedure
\end{algorithmic}
Loop Invariant: At the start of the $\textbf{i}-th$ iteration, the sub array $A[1....i-1]$ does not contain v.\\
Proof:\\
Initialization\\
At $i = 1$ we have an empty sub array, so the condition is obviously satisfied.\\
Maintenance \\ 
Suppose the invariant is satified at the start of the $i-th$ iteration, we can have:\\
\begin{itemize} 
	\item $A[i] = v$, A[i....i-1] does not contain v, and we can go to termination
	\item  $A[i] \neq v$ , $A[1....i]$ does not contain v
\end{itemize}
Termination 
\begin{itemize}
	\item $A[i] = v$, $A[1....i-1]$ does not contain v, termination
	\item We got to $i = A .length + 1$ and didn't find v. $A[1.....A.length]$ does not have v and the procedure returns NIL, termination.
\end{itemize}
\end{document}