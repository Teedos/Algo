\subsection*{Groups of 7}
The number of elements greater than x will be at least:
\begin{equation*}
	4(\lceil \frac{1}{2} \lceil \frac{n}{7} \rceil \rceil -2) \geq \frac{2n}{7} - 8
\end{equation*}
Similarly, at least $\frac{2n}{7} - 8$. So the SELECT function will call itself recursively on step 5 on sub-problems of size (at most): $\frac{5n}{7} + 8$.\\

So we get the recurrence relation:
\begin{equation*}
	T(n) \leq \begin{cases} O(1), & if n \leq n_0\\ T(\lceil \frac{n}{7} \rceil) + T(\frac{5n}{7} + 8) + O(n), &if n \geq n_0\end{cases}
\end{equation*}
We can determine an upper bound for: $T(n) \leq cn$  and $O(n) \leq an$
\begin{equation*}
	T(n) \leq c(\lceil \frac{n}{7} \rceil) + c(\frac{5n}{7} + 8) + an \leq \frac{cn}{7} + c + \frac{5cn}{7} + 8c + an
\end{equation*}
\begin{equation*}
	= \frac{6cn}{7} + 9c + an
\end{equation*}
\begin{equation}\label{eq1}
	= cn - \frac{cn}{7} + 9c + an \leq cn
\end{equation}
\begin{equation*}
	= O(n)
\end{equation*}
Naturally, in order for \ref{eq1} to hold, we need $- \frac{cn}{7} + 9c + an \leq 0$.\\
Getting:
\begin{equation*}
	c \geq \frac{7an}{n-63}
\end{equation*}
If we assume that $n \geq n_0 = 126$, we have $\frac{n}{n - 63} \leq 2$.\\
So choosing $c \geq 14a$ will satisfy the inequality.
\subsection*{Groups of 3}
The number of elements greater than x will be at least:
\begin{equation*}
	2(\lceil \frac{1}{2} \lceil \frac{n}{7} \rceil \rceil -2) \geq \frac{n}{3} - 4
\end{equation*}
Similarly, at least $\frac{n}{3} - 4$. So the SELECT function will call itself recursively on step 5 on sub-problems of size (at most): $\frac{2n}{3} + 4$.\\
We get the following recurrence relation:
\begin{equation*}
	T(n) \leq T(\lceil \frac{n}{3} \rceil) + T(\frac{2n}{3} + 4) + O(n)
\end{equation*}
We assume that $T(n) \geq cnlogn$ and bound O(n) with an.
\begin{equation*}
	T(n) >=  T(\lceil \frac{n}{3} \rceil) + T(\frac{2n}{3} + 4) + an
\end{equation*}
\begin{equation*}
	= c(\frac{n}{3})log(\frac{n}{3}) + cnlogn+ c(\frac{2n}{3})log(\frac{n}{3}) + an
\end{equation*}
\begin{equation*}
	= c(\frac{n}{3})log(\frac{n}{3}) + cnlogn+ c(\frac{2n}{3})log(\frac{n}{3}) + an
\end{equation*}
\begin{equation*}
	T(n) \geq 2cnlogn + an 
\end{equation*}	
We have demonstrated that  $T(n) = \Omega(nlogn)$.