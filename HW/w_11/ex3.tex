\subsection*{b}
\subsubsection*{1}

\begin{equation*}
	T(n) = 2T(\frac{n}{2}) + cn 
\end{equation*}
$a = 2$, $b = 2$, $f(n) = cn$.\\
\begin{equation*}
	n^{log{b}a} = n
\end{equation*}
We have $fn = cn$, so for the second case of the master theorem:
\begin{equation*}
	T(n) =\Theta(nlogn)
\end{equation*}

\subsubsection*{2}
We assume N = n.
\begin{equation*}
	T(n) = 2T(\frac{n}{2}) + cn +2\Theta(n) = 4n + cn + 2c(\frac{n}{2}) + 4T(\frac{n}{4}) =
\end{equation*}
We can apply the geometric series:
\begin{equation*}
	= \sum_{i = 0}^{logn -1}(cn + 2^in) = cnlogn + n\frac{1 - 2^{logn} }{1 - 2} = cnlogn + n^2 - n =
\end{equation*}
\begin{equation*}
	= \Theta(n^2)
\end{equation*}

\subsubsection*{3}
\begin{equation*}
	T(n) = 2T(\frac{n}{2}) + cn  + n = 2T(\frac{n}{2} + (c + 1)n
\end{equation*}
$a = 2$, $b = 2$, $f(n) = (c + 1)n$.\\
\begin{equation*}
	n^{log{b}a} = n
\end{equation*}
We have $fn = (c + 1 )n$, so for the second case of the master theorem:
\begin{equation*}
	T(n) =\Theta(nlogn)
\end{equation*}